\documentclass[a4paper]{article}

\usepackage{seestyle}

\begin{document}

\title{See Language Specification}
\author{Mihai B.}
\date{2010 June 16}
\maketitle

\section{Overview}
See (pronounced like the German word) is a simple procedural C-like language.
It has primitive types, structs, arrays and functions.
Variables of type \verb=struct= and \emph{array} are references.
Arrays are Java-like.

At the global level we have functions, struct definitions,
and variable declarations without initialization expressions.
All global variables are initialized to the default value
before starting program execution.
All local variables are initialized at the point of declaration:
with their initialization expression if present,
otherwise with the default value.
The default value is \verb=false= for \emph{booleans},
\verb=0= for other primitive types
and \verb=null= for references (arrays and structs).

\verb=struct=s contain only declarations:
for fields and for nested \verb=struct=s.
Nested \verb=struct=s can be used with the syntax \verb=Outer.Inner=.
Function bodies are code blocks.
A code block can contain statements, \verb=struct= declarations,
local variable declarations and nested code blocks (with the same elements).

\subsection{Notes}
The language does not allow declarations without definitions
(i.e.\ declaring a variable without an initilization expression means
initializing it with the default value for its type).
This document uses the terms \emph{declaration} and \emph{definition}
interchangeably; they both mean \emph{definition}.

The word \emph{reference} is used with two meanings in this text:
referencing a symbol (e.g.\ referencing \verb=x= in the expression \verb=x+1=)
and a variable of type \verb=struct= or array.
The author has tried to make context indicate the intended meaning
(sometimes replacing the word with \emph{variable}).


\section{Scopes}
A See program has a global scope.
Functions, nested code blocks and \verb=struct=s all introduce new scopes.

Instructions \emph{define} and \emph{reference} symbols.
Expressions \verb=x+2= and \verb=f(5)=
reference variable \verb=x= and function \verb=f=.
A variable declaration (e.g.\ \verb=A a;=) must first reference type \verb=A=
then define variable \verb=a= in the current scope.
A \verb=struct= declaration and a function declaration both
define new symbols in the current scope (a new type and a new function).
The function declaration also needs to reference
the function's return type and its formal parameter types.


\subsection{Resolving References}
In general, symbol references are resolved starting in the current scope
and moving outward (up to and including the global scope)
until a symbol with the referenced name is found.
When resolving a \emph{field access} (or \emph{nested struct}) expression
such as \verb=head.next=,
\verb=next= is resolved \emph{exclusively} within that \verb=struct='s scope
(e.g.\ \verb=Node head;=).

\paragraph{Forward references}
are generally allowed.
The only exception are forward references to local variables, which are illegal
(like in Java and C++).
A detailed discussion follows.

Functions can be forward-referenced.

Global variables can be referenced from any function
(even if the function is placed before the variables in the input file).
Global variables can not have initializing expressions,
and are initialized to the default values
(\verb=0=, \verb=false= or \verb=null=)
before starting program execution.

Types (whether defined globally or elsewhere) can be forward-referenced
(as long as they are visible in the scope where they are being used).

Local variables (defined in code blocks) may only be referenced
\emph{after} their point of declaration.


\subsection{Redefining Symbols}

Declarations in code blocks (i.e.\ local variables and types)
can redefine only global symbols; they can't redefine
function parameter names or symbols defined elsewhere in the function
(i.e.\ other local variables and types).
Declarations in \verb=struct= bodies (i.e.\ field names and nested type names)
have practically no limitations
(they must not collide with each other
or with the containing \verb=struct='s name).

\begin{program}
\begin{verbatimtab}
struct A {
	int x;
	struct B {
		int y;
		struct ?name? { .. }
		float ?name?;
	}
}
\end{verbatimtab}
\caption{Struct scopes: restrictions for contained names%
	\label{scopes:struct-decl}}
\end{program}

Using Listing \ref{scopes:struct-decl} as an example,
we'll show what names can be used inside a \verb=struct= declaration
(replacing \emph{?name?} in the example).
\verb=struct B= happens to be declared inside \verb=struct A=
in Listing \ref{scopes:struct-decl},
but it can be declared at the top level or inside a function:
the following restrictions for \verb=B='s body stay the same.

\emph{?name?} can be anything except \verb=B= and \verb=y=.
In particular, \emph{?name?} can be \verb=A= or \verb=x=
or the same name as a function.

\begin{program}
\begin{verbatimtab}
int f(A p) {
	A a;
	struct A {
		int i;
	}
	return a.i + p.x;
}
\end{verbatimtab}
\caption{Function scopes: defining local variables and types%
	\label{scopes:func-decl}}
\end{program}

Consider Listing \ref{scopes:func-decl} in the same file as
Listing \ref{scopes:struct-decl}.
This is an example of redefining symbols inside a function.

Functions introduce two (nested) scopes:
an (outer) scope for their formal parameters,
and another (inner one) for their body.
In the outer (parameter) scope we have \verb=p1=,
of type \verb=A= from Listing \ref{scopes:struct-decl}.
Another \verb=struct A= is defined in the function's body, 
so local variable \verb=a= has this new type
(which hides the outer \verb=A= in the function's body).
Variable \verb=a='s definition forward references local type \verb=A=.
The type is visible everywhere inside it's enclosing block.

If we define a symbol \verb=x= somewhere in a code block,
all instructions in that code block resolve \verb=x= to this local symbol.
If \verb=x= is a type, it can be forward-referenced.
If it's a local variable, forward references are illegal.

When defining a symbol in a code block,
the symbol's name must be different from all of the following:
\begin{itemize}
\item the enclosing function's name
\item the enclosing function's formal parameter names
\item all local variables visible from the current code block
\item all local types (\verb=struct=s) visible from the current code block
\end{itemize}
In other words, we can redefine only global symbols
with local type and variable names.

\end{document}
